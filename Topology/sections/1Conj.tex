\section*{Conjuntos}

\- \hrule  
\begin{proposition}
  \textbf{Ejercicio} \emph{(Leyes de De Morgan.)} \(\displaystyle X\backslash \bigcup A_i = \bigcap X\backslash A_i\) \ \ y \ \ \(\displaystyle X \backslash\bigcap A_i = \bigcup X\backslash A_i\). 
\end{proposition}

\begin{exercise}
  La imagen inversa es bien portada, con uniones, intersecciones y complementos. 
\end{exercise}

\begin{exercise} {\small
  \((a) \ \displaystyle f\left(\bigcup A_i\right) = \bigcup f(A_i); \ (b)\  f\left(\bigcap A_i\right) \subset \bigcap f(A_i)\ \footnote{Igualdad si \(f\) inyecta.}; \ (c)\ \overset{\hookrightarrow}{f}(X \backslash A) \subset Y \backslash f(A); \ (d) \  \overset{\twoheadrightarrow}{f}(X\backslash A) \supset Y\backslash f(A) .\)  
  }
\end{exercise}

\begin{proposition}
  \textbf{Ejercicio} \(\displaystyle \ (a)\  A\subset f^{-1}(f(A))\text{ igualdad si } \overset{\hookrightarrow}{f}; \ (b) \  f(f^{-1}(B)) \subset B \text{ igualdad si } \overset{\twoheadrightarrow}{f}\). 
\end{proposition}
\hrule 
