% PROGRAMA DE PÓS-GRADUAÇÃO EM ECONOMIA APLICADA
% UNIVERSIDADE FEDERAL DO RIO GRANDE - FURG

% ===========================================================

\documentclass[a4paper, 11pt]{article}

%%%%%%%%%%%%%%%%%%%%%%%%% PACOTES %%%%%%%%%%%%%%%%%%%%%%%%%%%%%%%%%%
 
\usepackage[margin = 0.5in, bottom = 1.2in]{geometry}
\usepackage[utf8]{inputenc}
\usepackage[T1]{fontenc}
\usepackage{graphicx}
\usepackage{xcolor}
\usepackage{lipsum}	
\usepackage{amsfonts,amsmath,amssymb, amsthm}
\usepackage{booktabs}
\usepackage{hyperref} 
\usepackage{dsfont}

% Configuração da Fonte
\usepackage{times} % Fonte: Times New Roman

% PACOTES ESSENCIAIS
\usepackage{caption}  

%\setlength{\footskip}{0pt} % Espacio entre texto y pie de página
\renewcommand{\footnoterule}{\vspace{5pt}\hrule width 0.3\linewidth\vspace{5pt}}
\setlength{\textheight}{770pt}
\setlength{\headsep}{10pt}
\setlength{\topmargin}{-60pt}

\begin{document} %Início do Documento

%\layout
%\newpage 
\-\vspace{-0.3in} 

\par\noindent \includegraphics[width = \textwidth]{Imagens/Header.png} 

%%%%%%%%%%%%%%%%%%%%%%%% CONFIG. DO TÍTULO %%%%%%%%%%%%%%%%%%%%%%%%

\begin{center} %Inicia a centralização
{\Large\textbf{Análisis en \(\mathbb{R}^n\)\ \ | \ \  2025-I}} \vspace{0.1in}   \\ {\small
Nestor Heli Aponte Avila\(^1\) \\ 
\href{mailto:n267452@dac.unicamp.br}{\url{n267452@dac.unicamp.br}}} %\\
%Prof. Piper Pimienta \\
%\href{mailto:pipe@dac.unicamp.br}{\url{pipe@dac.unicamp.br}}}

\end{center} %Finaliza a centralização

%%%%%%%%%%%%%%%%%%%%%%%%%  MEU PREAMBULO %%%%%%%%%%%%%%%%%%%%%%%%%%%%%%%%%

\theoremstyle{definition}
\newtheorem*{definition}{\rotatebox{45}{\large \(\square\)}}

\theoremstyle{plain}
\newtheorem*{lemma}{{\scriptsize \(\square\)}}
\newtheorem*{proposition}{{\large \(\square\)}}
\newtheorem*{theorem}{{\large \(\blacksquare\)}} 

\makeatletter
% Eliminar la puntuación del estilo 'plain' y 'definition'
\def\@thm@headpunct{} % Elimina el punto de los títulos de los teoremas

% Modificar el estilo de 'plain' y 'definition'
\def\th@plain{%
  %\thm@headfont{\normalfont\bfseries} % Mantener la fuente normal, sin negrita
  %\thm@notefont{\normalfont} % El texto de la nota (si la hay) no será en negrita
  \thm@headpunct{} % Sin puntuación adicional 
}

\def\th@definition{%
  %\thm@headfont{\normalfont\bfseries} % Fuente normal para el título
  %\thm@notefont{\normalfont} % El texto de la nota (si la hay) no será en negrita
  \thm@headpunct{} % Sin puntuación adicional
}

\def\th@remark{%
  %\thm@headfont{\normalfont\bfseries} % Fuente normal para el título
  %\thm@notefont{\normalfont} % El texto de la nota (si la hay) no será en negrita
  \thm@headpunct{} % Sin puntuación adicional
}
\makeatother

\theoremstyle{remark}
\newtheorem*{example}{Ejemplo}
\newtheorem*{exercise}{Ejercicio}
\newtheorem*{note}{\(*\)}

\newcommand{\R}{\mathbb{R}}
\newcommand{\E}{\vspace{0.2in}}
\newcommand{\N}{\mathbb{N}}
\newcommand{\U}[3]{\(#1 : U^{ab} \subseteq \R^{#2} \rightarrow \R^{#3}\)}
\newcommand{\ab}{{\rotatebox{0}{\scriptsize$\lor$}}}
\newcommand{\ce}{{\text{\ \large \rotatebox{90}{$\triangleleft$}}}}

%%%%%%%%%%%%%%%%%%%%%%%%% CONTENIDO %%%%%%%%%%%%%%%%%%%%%%%%%%%%%%%%%%

\section*{Aula 06/03/25}

\begin{definition}%[\textbf{Derivada Direccional}] 
    \textit{Derivada Direccional.} Sea \(U^{ab}\subseteq \R^n \) y \(f:U\rightarrow \mathbb{R}\) una función escalar. Para \(\vec{v}\in \mathbb{R}^n\) y \(p\in U\) definimos la derivada de \(f\) con dirección \(v\) en el punto \(p\) como:
    \begin{equation*}
    \frac{\partial f}{\partial v}(p) := \lim\limits_{h\to 0} \frac{f(p+hv) - f(p)}{h} \hspace{0.3in} \sim \hspace{0.3in} f(x+hv) = f(p) + \frac{\partial f }{\partial v}(p)\cdot h + \sigma(h) \ \footnote{Si existe \(\alpha \in \R \) tal que \(f(p+hv) = \alpha h + \sigma(h)\), entonces \(\frac{\partial f }{\partial v}(p) = \alpha\).}.
    \end{equation*}
    Si \(v = e_i\) entonces la llamamos $i$-ésima derivada o \textit{derivada parcial} y la denotamos por \(\frac{\partial f}{\partial x_i}(p)\). 
  \end{definition}

\vspace{0.2in}

\hrule
  \begin{example}
    \(f:\R^2\rightarrow \R \) dada por \(f(\textbf{x}) = \|\textbf{x}\|^2\). Hint: \(\alpha\).  
  \end{example}
  \begin{exercise}
    Sean \(\lambda\in \R\) y \(\vec{v},\vec{\omega} \in \R^n\). Muestre que \(\frac{\partial f }{\partial \lambda v} (p) = \lambda \frac{\partial f }{\partial v}(p)\). ¿Vale en general \(\frac{\partial f }{\partial (v+\omega)}(p) = \frac{\partial f }{\partial v}(p) + \frac{\partial f }{\partial \omega}(p )\)? 
  \end{exercise}
  \begin{example}
    \(g :\R^2\backslash\{0\}\rightarrow \R\) tal que \(g(x,y) = \arctan (y/x)\). Hint: voleo. 
  \end{example}
  \hrule 

 \vspace{0.2in}

  \begin{note}
    En un punto, la existencia de todas las derivadas parciales no implica que exista la derivada en toda dirección y tampoco implica continuidad de \(f\) en \(p\). 
  \end{note}

\vspace{0.2in}

  \hrule
  \begin{example}
    \(\displaystyle f : \R^2\rightarrow \R\) dada por \(\displaystyle f(x) = \begin{cases}
      x+y, &\text{si\ }x=0\ \text{o}\ y=0\\
      0, &\text{e. o. c.}
    \end{cases}\). Hint: direcciones. 
  \end{example}
  \begin{example}
    \(\displaystyle f:\R^2\rightarrow \R\) dada por \(\displaystyle f(x) = \begin{cases}
      \frac{xy^2}{x^2+y^2}, &\text{si\ }(x,y)=0\\
      0, &\text{e. o. c.}
    \end{cases}\). Hint: continuidad. 
  \end{example}
  \begin{exercise}
    Muestre que la existencia de derivadas parciales acotadas en todo punto \(p\in U\) implica \(f\) constante en \(U\). 
  \end{exercise}
  \hrule

  \vspace{0.2in}

  \begin{definition} 
    \textit{Conexidad.} Sea \(X^{ab}\subseteq \R^n\), \(X\) es conexo sii no existen \(U^{ab},V^{ab}\subseteq \R^n\) disjuntos, tales que su intersección con \(X\) es no vacía y \(X\subseteq U\cup V\). 
  \end{definition}

\vspace{0.2in}

  \hrule 
  \begin{proposition}
    \textbf{Ejercicio} \ Sea \(U^{ab}\subseteq \R^n\) conexo. Dados \(p,q\in U\) existe un camino polígonal en \(U\) con vértices
    \[p = p_0, p_1, \ldots, p_k = q,\]
    tal que \(p_{i+1}- p_i\) es colineal con algún \(e_j\) para cada \(0\leq i\leq k-1\) y algún \(j \in \{1,\ldots,n\}\). \footnote{El lema solo vale cuando \(U\) es abierto, piense en \(\mathds{S}^1\).\vspace{0.5in}}
  \end{proposition}
  \hrule 

\vspace{0.2in}


  \begin{proposition}
    Sea \(U^{ab}\subseteq \R^n\) conexo y \(f:U\rightarrow \R\) una función tal que \(\frac{\partial f}{\partial x_i} = 0\) en \(U\) para cada \(1\leq i\leq n\), entonces \(f\) es constante.  
  \end{proposition}

  \vspace{0.2in}
\section*{Aula 11/03/25}

\begin{definition}
    \textit{Derivadas de orden superior.} Sea \(f: U^{ab}\subseteq \R^n \rightarrow\). Si existe \(\frac{\partial f }{\partial x_i}(p)\) para cada \(p\in U\) podemos definir \(\frac{\partial f}{\partial x_i}: U \rightarrow \R\) y considerar la \emph{derivada de segundo orden}
    \[\frac{\partial^2 f}{\partial x_j \partial x_i}(p) := \frac{\partial}{\partial x_j}\left(\frac{\partial f}{\partial x_i}\right)(p).\]
\end{definition}

\begin{note}
    El orden importa, en general. 
\end{note}

\begin{definition}
    Sea \(k\in \{0\}\cup \N \). Decimos que \(f:U\rightarrow \R\) es de clase \(C^k(U)\) si para cada \(m \leq k\) todas las derivadas de orden \(m\) de \(f\) existen y son continuas en \(U\). La función \(f\) es \emph{smooth} en \(U\) si es de clase \(C^\infty(U)\).\footnote{\(C^0(U)\) es la clase de las funciones continuas en \(U\).} 
\end{definition}

\vspace{0.2in}

\hrule
\begin{example}
    Los elementos de \(\R[x_1,\ldots,x_n]\), el anillo de polinomios de \(n\) variables con coeficientes en \(\R\) es de clase \(C^\infty\). 
\end{example}
\begin{example}
    \(\det(T): M_{n\times n}(\R) \cong \R^{n^2}\rightarrow \R\) es de clase \(C^\infty\). 
\end{example}
\begin{example}
    \(x\mapsto x^{1/3}\) es de clase \(C^0\) más no es de clase \(C^1\) en \(0\). 
\end{example}
\begin{exercise}
    Muestre que \(C^k(U)\) es una \(\R\)-álgebra conmutativa, con la suma y el produto usual de funciones. Además de es eso pruebe que 
    \[C^0(U)\supsetneq C^1(U)\supsetneq \cdots \supsetneq C^\infty(U), \]
    Y note que los elementos inversos son aquellos que no se anulan. \footnote{Colorario de esto es que las funciones racionales son \(C^\infty(U) \) en dominios donde no se anulan.} 
\end{exercise}
\begin{example}
    \(\displaystyle f(x,y) = \begin{cases}
        \frac{xy(x^2+y^2)}{x^2+y^2}, \ &\text{si}\ (x,y)\neq 0 \\
        0, \ &\text{e.o.c.}
    \end{cases}\). Hint: averigue si esta en \(C^2\)
\end{example}
\hrule 

\vspace{0.2in}

\begin{theorem}[Schwarz]
    \itshape Sea \(f:U^{ab}\subseteq \R^n \rightarrow \R\) una función de clase \(C^2(U)\), entonces para cada \(1\leq i,j\leq n\),  
    \[\frac{\partial^2 f}{\partial x_i \partial x_j} = \frac{\partial^2 f }{\partial x_j \partial x_i}.\]
\end{theorem}

\vspace{0.2in}

\hrule 
\begin{exercise}
    \emph{Colorario.} Si \(f\in C^k(U)\) entonces no importa el orden en que son tomadas las derivadas de orden \(m\leq k\). 
\end{exercise}
\hrule 

\vspace{0.2in}

\begin{definition}
    \emph{Diferenciabilidad}. Decimos que una función \(f:U^{ab}\subseteq \R^n \rightarrow \R\) es diferenciable en un punto \(p\in U \) si existe un funcional lineal \(\ell : \R^n \rightarrow \R\) tal que 
    \[f(p+v) = f(p) + \ell\cdot v + \sigma(\|v|\|)\ \text{cuando} \ v\to 0. \]
\end{definition}
\begin{proposition}
    Si \(f\) es diferenciable en \(p\) entonces para cada \(\vec{v_0}\in \R^n\), \(\ell \cdot v = \frac{\partial f }{\partial v}(p)\). En particular \(\ell_i = \frac{\partial f}{\partial x_i}(p)\).\footnote{Esto también implica que las derivadas direccionales son lineales en esa dirección.}
\end{proposition}
\begin{definition}
    \emph{Diferencial.} Si \(f\) es diferenciable en \(p\) el diferencial de \(f\) en \(p\) es una función lineal \(df(p): \R^n \rightarrow \R\) dada por 
    \[df(p)\cdot v := \sum_{i=1}^n \frac{\partial f}{\partial x_i}(p)\cdot v_i.\]
\end{definition}
\begin{note}
    La existencia de derivadas direccionales sirve para definir \(df(p)\), sin embargo esto no garantiza que sea lineal y aún si lo fuera no es garantía de diferenciabilidad en el punto. 
\end{note}

\vspace{0.2in} 
\section*{Aula 13/03/25}
\-\hrule 
\begin{example}
    \(f(\textbf{x}) = \|x\|^2\). Hint: separe y contemple. 
\end{example}
\begin{exercise}
    El diferencial de una función lineal es él mismo. 
\end{exercise}
\hrule

\vspace{0.2in}

\begin{proposition}
    Si \U{f}{n}{} es diferenciable en \(p\) entonces es continua en \(p\). 
\end{proposition}

\vspace{0.2in} 

\hrule 
\begin{example}
    \(\displaystyle f(x,y) = \begin{cases}
        \frac{xy^2}{x^2+y^2}, \ &\text{si} \ (x,y)\neq 0 \\
        0, \ &\text{e. o. c.}
    \end{cases}\). Hint: verifique derivadas, analice si \(f(v) = \sigma(\|v\|)\). \footnote{Si la derivada direccional en un punto no es lineal (coordenadas \(v_i\)) entonces el diferencial en ese punto no existe. }
\end{example}    
\begin{exercise}
    Estudie la diferenciabilidad de \(\displaystyle f(x,y) = \begin{cases}
        \frac{xy^3}{x^2+y^2}, \ &\text{si}\ (x,y)\neq 0 \\
        0, \ &\text{e.o.c}
    \end{cases}\)
\end{exercise}
\hrule 

\vspace{0.2in} 

\begin{note}
    Existencia de derivadas direccionales NO implica diferenciabilidad. 
\end{note}
\begin{theorem}
    Si \U{f}{n}{} es de clase \(C^1(U)\) entonces es diferenciable en cada punto de \(U\).\footnote{\(C^1 \Rightarrow C^0 \land \exists \frac{\partial f}{\partial v}\). }
\end{theorem}

\vspace{0.2in} 

\hrule 
\begin{example}
    Toda función polinomial es diferenciable. 
\end{example}
\begin{example}
    La proyección $i$-ésima \(x_i: \R^n\rightarrow \R\) es diferenciable. Además, puesto que es lineal \(dx_i(p) = x_i \) para cada \(p\in \R^n\). En particular notamos que \(dx_i(p)\cdot e_j = \delta_{ij}\), osea que \(\{dx_1(p), \ldots,dx_n(p)\}\) es una base para \({(\R^n)}^*\) \emph{(espacio dual)}\footnote{\(dx_i\) mide los incrementos de las variables independientes y los relaciona con los de la variable dependiente, osea \(df\).}. Suponiendo que pasa para cada \(p\in U\) el diferencial de \(f\) se escribe de forma única como 
    \[df = \sum_{i=1}^n \frac{\partial f }{\partial x_i} dx_i. \]
\end{example}
\begin{example}
    \(\Theta(x,y) = \arctan (y/x) \) definida para \((x,y)\neq 0\). Hint: Cálcule e interprete \(d\Theta\). 
\end{example}
\begin{proposition}
    \textbf{Ejercicio} \ Si \U{f,g}{n}{} son diferenciables, entonces \(f+g,\ fg\) y \(f/g\) (para \(g\neq 0\) en \(U\)) son funciones diferenciables en \(U\) y 
    \[d(f+g) = df + dg\ \ ;\ \ d(fg) = f\cdot dg + df\cdot g\ \ ; \ \ d\left(\frac{f}{g}\right) = \frac{df\cdot g - f\cdot dg}{g^2}\] 
\end{proposition}
\hrule 

\vspace{0.2in}

\begin{definition}
    \textit{Gradiente.} El gradiente de una función diferenciable en \(p\) es el vector compuesto por las derivadas parciales,
    \[\nabla f(p) := \left(\frac{\partial f}{\partial x_1}(p),\ldots,\frac{\partial f}{\partial x_n}(p)\right). \]
\end{definition}

\begin{note}
    Interpretando la definición tenemos que \(\langle \nabla f , v\rangle = \frac{\partial f }{\partial v}\). En este sentido sea \(\vec{u}\in \R^n\) un vector unitario, entonces 
    \[\Big|\frac{\partial f}{\partial u}(p)\Big| = |\langle \nabla f (p), u\rangle| \leq \|\nabla f(p)\|\|u\|,\]
    es decir, el gradiente apunta en la dirección de mayor crecimiento de \(f\) en \(p\).\footnote{Esta es la base de lo que se conoce como método del gradiente, para minimizar una función.} 
\end{note}

\vspace{0.2in}

\hrule 
\begin{exercise}
    Muestre que si \U{f}{n}{} es diferenciable y \(f(p)\) es un extremo local, entonces \(\nabla f (p) = 0\).  
\end{exercise}
\hrule 

\vspace{0.2in}

\section*{Aula 18/03/25}

\begin{proposition}
    Sean \(f:U^{ab} \rightarrow \R \), \(p\in U\) y \(\vec{v} \in \R^n\) tales que \([p,p+v]\subseteq U\). Considere \(\varphi(t) = f(p+tv)\). Si para algún \(t\in (0,1)\) \(f\) es diferenciable en \(p+tv\) entonces \(\varphi\) es diferenciable en \(t\) y \(\varphi\prime (t) = df(p+tv)\cdot v= \frac{\partial f }{\partial v} (p+tv)\). 
\end{proposition}
\begin{note}
    TVM implica que si \(f\) es continua en \([p,p+v]\) y diferenciable en \((p,p+v)\) entonces existe \(\theta \in (0,1)\) tal que \(f(p+v)-f(p)= df(p+\theta v )\cdot v = \frac{\partial f}{\partial v}(p+\theta v)\). 
\end{note}
\begin{definition}
    Sea \(\ell: R^n\rightarrow \R\) un operador lineal. La norma del operador \(|\ell | := \sup \|\ell v\| \text{ tales que } v\in \mathbb{S}^n\).  
\end{definition}

\E

\hrule
\begin{proposition}
    \textbf{Ejercicio} Muestre que \(\exists \vec{w}\in \R^n\) tal que \(\ell v = \langle w, v \rangle \). 
\end{proposition}
\hrule 

\E 

\begin{note}
    Del ejercicio resulta inmediato que \(|\ell | = \|w\|\). En particular, si \(f\) es diferenciable en \(p\), entonces \(|df(p)|= \|\nabla f (p)\|\). 
\end{note}
\begin{proposition}
   Sea \(K\subset U\) dominio convexo diferenciable de \(f\) y \( c\geq 0 \) tal que \(\|df(x)\|\leq c\) para cada \(x\in K\). Entonces para cada \(p,q\in K \) vale que \(\|f(p)-f(q)\|\leq c \|p-q\|\). 
\end{proposition}
\begin{definition}
    \(f:X\subset \R^n\rightarrow \R \) es \emph{Lipschitz continua} si existe \(c\geq 0\) tal que \(\|f(p)-f(q)\|\leq c \|p-q\|\) para cada \(p,q \in X\). 
\end{definition}
\begin{note}
    Osea que TVM también implica \(f\) Lipschitz. 
\end{note}

\E

\hrule
\begin{exercise}
    Muestre que si \(f\in C^1(U)\) y \(K\) es compacto y convexo, entonces la hipótesís de la proposición anterior es automaticamente satisfecha para algún \(c\geq 0\). 
\end{exercise}
\hrule 

\E

\subsection*{Formula de Taylor}

\begin{note}
    La idea es aproximar funciones por polinomios, apuntando a una forma \(f(p+v) = P_k[v] + \Gamma_k[v]\). Un polinomio en variable \(v\) de orden \(k\) y un error del orden \(\sigma (\|v\|^k)\)
\end{note}

\E

\hrule
\begin{example}
    Si \(f\in C^2\) entonces \( f(p+v) = \underbrace{f(p) + \sum \frac{\partial f }{\partial x_i}(p)v_i+ \frac{1}{2}\sum \frac{\partial f}{\partial x_i \partial x_j}(p) v_iv_j}_{P_2[v]} + \underbrace{\sigma(\|v\|^2)}_{\Gamma_2[v]}\)
\end{example}
\hrule 

\E 

\begin{definition}
    Para \(k\geq 2\), decimos que \(f\) es \(k\)-diff en \(p\) si existe \(B(p;\epsilon)\) donde existe \(df(p)\) y \(\frac{\partial f }{\partial x_i}\) es \((k-1)\)-diff. 
\end{definition}

\E 

\hrule 
\begin{example}
    \(f \in C^k(U)\)  implica que \(f\)  es \(k\)-diff en cada \(p\in U\). 
\end{example} 
\begin{exercise}
    Muestre que el Teorema de Schwarz vale en general para funciones \(2\)-diff. Hint: Elon Musk. 
\end{exercise}
\hrule 

\E

\begin{definition}
    Sea \(f\) una función \(k\)-diff en \(p\) y \(d^kf(p): \R^n \rightarrow \R\) tal que \(v\mapsto d^kf(p)v^{\otimes k} = \sum \frac{\partial^k f }{\partial x_{i_1}  \cdots \partial x_{i_k}}(p)v_{i_1}\cdots v_{i_k}\). \footnote{Función polinomial y homógenea de grado \(v_1\cdots v_k\)}
\end{definition}

\E \newpage 

\hrule 
\begin{example}
    Para \(f:\R^2\rightarrow \R\) tenemos: \( d^2f(p)v^{\otimes 2} = \sum\frac{\partial^2 f}{\partial x^2}(p)h^2 + 2\sum \frac{\partial^2 f }{\partial x\partial y}(p)hk+ \sum\frac{\partial^2 f}{\partial y^2}(p)k^2\). \footnote{Una forma cuadrática de grado 2.} 
\end{example}
\hrule 

\E

\begin{definition}
    Sea \(f\) \(2\)-diff en \(p\) entonces la Hessiana de \(f\) en \(p\) es \( Hf(p)= \left(\frac{\partial^2 f}{\partial x_i\partial x_j}(p)\right)_{i,j} =  \begin{pmatrix}
        \frac{\partial^2 f }{\partial x_1^2}(p) & \cdots &  \frac{\partial^2 f }{\partial x_1 \partial x_n}(p) \\
        \vdots & \ddots & \vdots \\
        \frac{\partial^2 f }{\partial x_n \partial x_1}(p) & \cdots & \frac{\partial^2 f }{\partial x_n^2}(p)
    \end{pmatrix}\)
\end{definition}
\begin{note}
    \(Hf(p)\) es única (Schwarz) y símetrica que representa a \(d^2f(p)v^{\otimes k} = \langle Hf(p)\cdot v, v\rangle = \displaystyle \sum_{|\alpha|=2} \binom{2}{\alpha}\partial^\alpha f(p)v^\alpha. \)
\end{note}

\E 

\hrule 
\begin{exercise}
    Muestre que si \(f\) es \(k\)-diff en \(p\) entonces \(d^kf(p)v^{\otimes k} =\displaystyle \sum_{|\alpha|=2} \binom{k}{\alpha}\partial^\alpha f(p)v^\alpha\). 
\end{exercise}
\hrule 
 


\section*{Aula 20/03/25}

\begin{definition}
    \(f\) se anula al orden \(k+1\) en \(p\) si \(\partial^\alpha f(p)=0\) para todo \(|\alpha|\leq k\). 
\end{definition}

\E 

\hrule 
\begin{exercise}
    Si \(f(x)= \sum c_\alpha x^\alpha \) se anula identicamente en una vecindad del origen entonces \(c_\alpha = 0\) para cada \(\alpha\). 
\end{exercise}
\hrule 

\E

\begin{theorem}
    Sean \(k\geq 1\) y \(f\) una función \(k\)-diff en \(0\in \R^n\). Si \(f\) se anula al orden \(k+1 \) entonces \(f(v)=\sigma(\|v\|^k)\).
\end{theorem}
\begin{proposition}
    \emph{Colorario.} Para \(p\in\R^n\) tenemos \(f(p+v)= \sum \frac{1}{i!}d^if(p) v^{\otimes i}  + \sigma(\|v\|^k)\). 
\end{proposition}    

\E

\hrule 
\begin{exercise}
    Demuestre el colorario. Hint: \(\Gamma_k(v)= \sigma(\|v\|)\)? 
\end{exercise}
\begin{exercise}
    Fórmula de Taylor infinitesimal de orden 5 de \(f(x,y) = \frac{x}{1+xy}\) en \(0\). Hint: Geometría - Taylor?  
\end{exercise}
\begin{exercise}
    Sea \(f\) función \(k\)-diff. Use la fórmula de Taylor para provar el reves del Teorema, concluya la unicidad. 
\end{exercise}
\hrule 

\E 

\begin{proposition}
    \emph{Resto de Lagrange e Integral.} Sea \(U\subseteq \R^n \) e \(f\in C^{k+1}(U)\), para cada \(p\in U\), \(\vec{v}\in \R^n\) tales que \([p,p+tv]\subseteq U\) tenemos: \((i) \  f(p+v) = T_k(v) + \frac{1}{(k+1)!} d^{k+1}f(p+\theta v) v^{\otimes (k+1)} = \ (ii)\ T_k(v) + \frac{1}{k!} \displaystyle\int_0^1 (1-t)^kd^{k+1}f(p+tv) v^{\otimes (k+1)} dt \). 
\end{proposition}

\section*{Aula 25/03/25}

\begin{definition}
    Sea \(f:U^\ab\subseteq \R^n \to \R\) diferenciable. Un punto \(p\in U\) es \emph{punto crítico} de \(f\) sii \(df(p)=0=\nabla f(p)\). 
\end{definition}

\E

\hrule 
\begin{example}
    \(f(x,y) = x^2 + 3y^4+ 4y^3 - 12y^2\). Hint: cálcule no sea flojo. 
\end{example}
\hrule 

\E

\begin{definition}
    La función \(f\) tiene un máximo (o mínimo) local en \(p\) si \(f(x) \leq f(p)\) en alguna vecindad de \(p\) \footnote{Más generalmente decimos \emph{extremo local}.}.  
\end{definition}
\begin{proposition}
    Extremo local  \(\Rightarrow \) punto crítico. 
\end{proposition}
\begin{note}
    El objetivo ahora es clasificar como máximo o mínimo o ninguno de los dos. 
\end{note}

\E

\hrule 
\begin{example}
    Sea \(f\) una forma cuadrática con Hessiana diagonal \(H= (\lambda_1, \ldots,\lambda_n)\). En Taylor orden $2$ tenemos \(f(p+v)= f(p) + df(p) + \frac{1}{2}\sum \lambda_i v_i^2 + \sigma(\|v\|^2)\). Entonces \(\lambda_i > 0 \) para cada \(i \Rightarrow \) mínimo local en \(p\); en cambio si \(\lambda_i <0\) para cada \(i \Rightarrow \) máximo local en \(p\). 
\end{example}
\hrule 

\E

\begin{definition}
    Sea \(A = A^T\in M_{n\times n}(\R)\). La forma cuadrática \(f: v\mapsto \langle Av, v\rangle\) es \emph{positiva} si \(f(v)>0\), \emph{negativa} si \(f(v)<0\) para todo \(\vec{v}\in \R^n\setminus \{0\}\) o indefinida e.o.c.
\end{definition}

\E

\hrule 
\begin{example}
    (i) \(v\mapsto \|v\|^2\) -- positiva y (ii) \(v\mapsto t^2-x^2-y^2-z^2\) -- indefinida. 
\end{example}
\hrule 

\E

\begin{theorem}
    Toda matriz símetrica posee base ortonormal de autovectores, es diagonalizable. 
\end{theorem}

\E

\hrule
\begin{exercise}
    Sea \(A=A^T\in M_{n\times n}(\R)\) e \(\lambda_1, \ldots, \lambda_n\) sus autovalores, muestre que \(A\) es positiva (o negativa) si \(\lambda_i>0 \) (o \(\lambda_i <0\)) para cada \(1\leq i\leq n\). 
\end{exercise}
\hrule 

\E

\begin{theorem}
    Sea \(f\) función \(2\)-diff en \(p\in U\) punto crítico. Si \(Hf(p)\) es definida positiva (o negativa) entonces \(p\) es un mínimo (o máximo) isolado local de \(f\) en \(U\). 
\end{theorem}
\begin{definition}
    Un punto crítico \(p\) de \(f\) es punto de silla sii \(Hf(p)\) tiene por lo menos un \(\lambda_i >0\) y un \(\lambda_j<0\). 
\end{definition}
\begin{note}
    Sea \(\lambda_1 > 0 \) y \(\lambda_2<0\) con autovectores \(w_1\) y \( w_2\), entonces \(f\) tiene mínimo local en la dirección de \(w_1\) y máximo local en la dirección de \(w_2\). Hint: \(\langle Hf(p)\cdot v, v\rangle \). 
\end{note}
\begin{definition}
   Un \emph{punto degenerado} es un punto crítico donde la Hessiana es singular, \(Hf(p)= 0\).  
\end{definition}

\E

\hrule
\begin{exercise}
    Sea \(f(x,y)= (y-x^2)(y-2x^2)\). Muestre que \(0\) es un punto degenerado. Muestre que la restricción de \(f\) a cualquier recta que pasa por el origen tiene mínimo local en \(0\) (sin embargo no lo tiene en \(f\)). Hint: considere conjunto donde \(f>0\) y \(f<0\). 
\end{exercise}
\hrule 
\section*{Aula 27/03/25}

\begin{theorem}
    Sea \(K\subseteq \R^n\) y \(f:K\to \R\) continua, entonces \(f\) tiene máximo y mínimo global en \(K\). 
\end{theorem}
\begin{note}
    Podemos agregar condiciones para que funcione aún en conjuntos no compactos. 
\end{note}

\E

\hrule
\begin{example}
    Sea \(f(x,y) = \frac{x}{x^2 + (y-1)^2 + 4 }\). Estudie máximos y mínimos de \(f\) en \(Q = \{(x,y) : x\geq 0 \text{ y } y\geq 0\}\). 
\end{example}
\hrule 

\E

\begin{proposition}
    Sea \(F^\ce\subseteq \R^n\) no acotado y \(f:F\to \R^n\) continua. Tenemos (i) Si \(f(x)\to \infty \) cuando \(\|(x,y)\|\to \infty \), entonces \(f\) tiene un mínimo global y (ii) Si \(f(x) \to 0\) cuando \(\|(x,y)\|\to \infty\), entonces \(f\) tiene un máximo global. 
\end{proposition}

\subsection*{Problems with Constraints - Optimización}

El objetivo aquí es mínimizar o máximizar funciones en conjuntos de la forma \(H = \{x\in \R^n: g(x)=0\}\). Si \(g\in C^k\) entonces \(H\) es una hipersuperficie de clase \(C^k\) definida por la ecuación \(g(x)=0\). 

\E

\hrule 
\begin{example}
    \(\mathbb{S}^{n-1} = \{x\in \R^n: \|x\| = 1\}\) definida por \(\left(\sum x_i^2\right) - 1 = 0 \). 
\end{example}
\begin{exercise}
    Verifique bajo que condiciones \(p\in \mathbb{S}^{n-1}\) implica \(dg(p)\neq 0\). 
\end{exercise}
\begin{note}
    La condición encontrada en el ejercicio anterior es garantía de que \(H\) es una variedad de clase \(C^k\) (sin singularidades) y que el espacio tangente de \(H\) en \(p\) es \(T_pH=\ker(dg(p))=\{v\in \R^n: dg(p)\cdot v = 0\}\).  
\end{note}
\begin{example}
    Máximos y mínimos de \(f(x,y) = x^2+y^2+y\) en el disco \(D=\{(x,y):x^2+y^2 \leq 1 \}\). Hint: Lagrange?. 
\end{example}
\hrule 

\E

\begin{theorem}
    \emph{Multiplicadores de Lagrange.} Sea \(f:U^\ab\subseteq \R^n\to \R\) de clase \(C^2\) y \(H=\{x\in \R^n: g(x)=0\}\) una superficie de clase \(C^1\) contenida en \(U\). Si \(p\in U\) es extremo local de \(H_H: H\to \R\), entonces \(\exists \lambda \in \R\) tal que \(df(p) =\lambda dg(p)\).   
\end{theorem} 

\E

\hrule
\begin{exercise}
    Demuestre el Teorema espectral vía Multiplicadores de Lagrange.     
\end{exercise}
\hrule 

%%%%%%%%%%%%%%%%%%%%%%%%%% REFERÊNCIAS %%%%%%%%%%%%%%%%%%%%%%%%%%%%%%%%

\begin{thebibliography}{9}

%\bibitem{martin1966complex}
%Martin, D. y Ahlfors, L.V. (1966). \textit{Complex Analysis}. New York: McGraw-Hill.

\bibitem{lima2004analise}
Lima, Elon Lages (2004). \textit{An{\'a}lise real Vol. 2}. IMPA, Rio de Janeiro.

\end{thebibliography}

%%%%%%%%%%%%%%%%%%%%%%%%%% APÊNDICES %%%%%%%%%%%%%%%%%%%%%%%%%%%%%%%%

\end{document}
