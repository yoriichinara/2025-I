\section*{Aula 20/03/25}

\begin{definition}
    \(f\) se anula al orden \(k+1\) en \(p\) si \(\partial^\alpha f(p)=0\) para todo \(|\alpha|\leq k\). 
\end{definition}

\E 

\hrule 
\begin{exercise}
    Si \(f(x)= \sum c_\alpha x^\alpha \) se anula identicamente en una vecindad del origen entonces \(c_\alpha = 0\) para cada \(\alpha\). 
\end{exercise}
\hrule 

\E

\begin{theorem}
    Sean \(k\geq 1\) y \(f\) una función \(k\)-diff en \(0\in \R^n\). Si \(f\) se anula al orden \(k+1 \) entonces \(f(v)=\sigma(\|v\|^k)\).
\end{theorem}
\begin{proposition}
    \emph{Colorario.} Para \(p\in\R^n\) tenemos \(f(p+v)= \sum \frac{1}{i!}d^if(p) v^{\otimes i}  + \sigma(\|v\|^k)\). 
\end{proposition}    

\E

\hrule 
\begin{exercise}
    Demuestre el colorario. Hint: \(\Gamma_k(v)= \sigma(\|v\|)\)? 
\end{exercise}
\begin{exercise}
    Fórmula de Taylor infinitesimal de orden 5 de \(f(x,y) = \frac{x}{1+xy}\) en \(0\). Hint: Geometría - Taylor?  
\end{exercise}
\begin{exercise}
    Sea \(f\) función \(k\)-diff. Use la fórmula de Taylor para provar el reves del Teorema, concluya la unicidad. 
\end{exercise}
\hrule 

\E 

\begin{proposition}
    \emph{Resto de Lagrange e Integral.} Sea \(U\subseteq \R^n \) e \(f\in C^{k+1}(U)\), para cada \(p\in U\), \(\vec{v}\in \R^n\) tales que \([p,p+tv]\subseteq U\) tenemos: \((i) \  f(p+v) = T_k(v) + \frac{1}{(k+1)!} d^{k+1}f(p+\theta v) v^{\otimes (k+1)} = \ (ii)\ T_k(v) + \frac{1}{k!} \displaystyle\int_0^1 (1-t)^kd^{k+1}f(p+tv) v^{\otimes (k+1)} dt \). 
\end{proposition}
