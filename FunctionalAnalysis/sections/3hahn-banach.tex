\section*{Teoremas de Hahn-Banach}

\begin{proposition}
    \emph{Lemma de Zorn.} Todo conjunto parcialmente ordenado, no vacío y en el cual todo subconjunto totalmente ordenado tiene cota superior, tiene un elemento máximal. 
\end{proposition}

\E

\hrule
\begin{exercise}
    Lemma de Zorn \(\Leftrightarrow \) Axioma de Elección. 
\end{exercise}
\hrule 

\E

%\begin{theorem}
%    \emph{Hanh-Banach (en \(\R\)).} Sean \(E\) un espacio (sobre \(\R\)) normado y \(p:E\to \R\) una función tal que, 
%    \begin{itemize}
%        \item \(\forall a>0,\forall x\in E\) se tiene \(p(ax) = ap(x)\). 
%        \item \(\forall x,y\in E\) su cumple \(p(x+y) \leq p(x)+p(y)\). 
%    \end{itemize} 
%    Sean tambien \(G\leq E \) y \(\varphi : G \to \R\) un operador lineal tal que \(\forall x\in G \text{ se tiene }\varphi(x)\leq p(x)\). Entonces \(\exists \overset{\sim }{\varphi} : E \to \R\) lineal que extiende a \(\varphi\), es decir, \(\overset{\sim }{\varphi}(x)\Big|_{G} = \varphi(x) \) y que además satisface \(\forall x\in E \text{ que }\overset{\sim}{\varphi}(x)\leq p(x)\). 
%\end{theorem}
\begin{theorem}
    \emph{Hanh-Banach (en \(\K\)).} Sean \(E\) un espacio (sobre \(\K\)) normado y \(p:E\to \R\) una función tal que, 
    \begin{itemize}
        \item \(\forall a>0,\forall x\in E\) se tiene \(p(ax) = |a|p(x)\). 
        \item \(\forall x,y\in E\) su cumple \(p(x+y) \leq p(x)+p(y)\). 
    \end{itemize} 
    Si \(G\leq E \) y \(\varphi : G \to \K\) es un operador lineal tal que \(\forall x\in G \text{ se tiene }|\varphi(x)|\leq p(x)\), entonces \(\exists \overset{\sim }{\varphi} : E \to \K\) lineal que extiende a \(\varphi\), es decir, \(\overset{\sim }{\varphi}(x)\Big|_{G} = \varphi(x) \) y que además satisface \(\forall x\in E \text{ que }|\overset{\sim}{\varphi}(x)|\leq p(x)\). 
\end{theorem}
\begin{note}
    \emph{Colorario(s).} \begin{itemize}
        \item Si \(\varphi \) es continuo entonces \(\overset{\sim}{\varphi}\) también y \(\|\varphi\| = \|\overset{\sim}{\varphi}\|\). 
        \item Si \(E\) es un espacio normado entonces \(\forall x_0\in E \setminus \{0\},\exists \varphi\in E'\) tal que \(\|\varphi\| = 1\) y \(\varphi(x_0) = \|x_0\|\). 
        \item Si \(E\neq \{0\}\) y \(x\in E \) entonces \(\|x\| = \sup\{ |\varphi(x)|: \varphi \in E' \text{ y } \|x\|\in B_E\} \) cuyo valor alcanza.   
    \end{itemize}
\end{note}